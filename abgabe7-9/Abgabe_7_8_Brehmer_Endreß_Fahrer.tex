\documentclass[accentcolor=tud10b,colorbacktitle,inverttitle,landscape,german,presentation,t]{tudbeamer}
\usepackage{ngerman}
\usepackage[utf8]{inputenc}
\usepackage{graphicx}
\usepackage{multirow}
\providecommand\thispdfpagelabel[1]{}

\begin{document}

\title[MLDM: Projekt Aufg 7-8]{Maschinelles Lernen Symbolische Ansätze:\\ Projekt Aufgaben 7-8}
\subtitle{}

\author[brehmer\_endreß]{Joachim F. Brehmer-Moltmann, Jeannine Endreß}
%\institute[]{}

\date{\today}

\begin{titleframe}
\tableofcontents
\end{titleframe}

    \section{Aufgabe 7 - Ensemble-Lernen}
    
    \subsection{Benutzte Datensätze}
    
    \begin{frame}[t]
    \frametitle{Aufgabe 7 - Ensemble-Lernen\\ Benutzte Datensätze}
        \begin{itemize}
            \item labor
            \item yeast
            \item car
        \end{itemize}
    \end{frame}
    
    \subsection{Regulärer J48, Bagging und AdaBoost}
    
    \begin{frame}[t]
    \frametitle{Aufgabe 7 - Ensemble-Lernen\\ Datensatz labor}
        \begin{tabular}{|c||c|c|c|c|c|c|c|c|c|c|}
        \hline 
        Anzahl Iterationen & 1 & 2 & 3 & 4 & 5 \\ 
        \hline 
        J48 & 73.7\% & / & / & / & / \\ 
        \hline 
        AdaBoost & 73.7\% & 77.2\% & 82.5\% & 80.7\% & 82.5\% \\ 
        \hline 
        Bagging & 71.9\% & 78.9\% & 77.2\% & 80.7\% & 82.5\% \\ 
        \hline 
        Random Forests & 84.2\% & 80.7\% & 84.2\% & 86.0\% & 87.7\% \\ 
        \hline 
        \end{tabular}
        
        \begin{tabular}{|c||c|c|c|c|c|c|c|c|c|c|}
        \hline 
        Anzahl Iterationen & 6 & 7 & 8 & 9 & 10 & 1000 \\ 
        \hline 
        J48 & / & / & / & / & / & /\\ 
        \hline 
        AdaBoost & 86.0\% & 84.2\% & 87.7\% & 87.7\% & 89.5\% & 87.7\% \\ 
        \hline 
        Bagging & 82.5\% & 86.0\% & 86.0\% & 84.2\% & 86.0\% & 86.0\% \\ 
        \hline 
        Random Forests & 87.7\% & 87.7\% & 87.7\% & 87.7\% & 87.7\% & 89.5\% \\ 
        \hline 
        \end{tabular} 
    \end{frame}  
    
    \begin{frame}[t]
    \frametitle{Aufgabe 7 - Ensemble-Lernen\\ Datensatz yeast}
        \begin{tabular}{|c||c|c|c|c|c|c|c|c|c|c|}
        \hline 
        Anzahl Iterationen & 1 & 2 & 3 & 4 & 5 \\ 
        \hline 
        J48 & 56.0\% & / & / & / & / \\ 
        \hline 
        AdaBoost & 56.0\% & 46.6\% & 53.8\% & 54.6\% & 56.3\% \\ 
        \hline 
        Bagging & 50.3\% & 51.2\% & 55.3\% & 57.1\% & 56.9\% \\ 
        \hline 
        Random Forests & 47.8\% & 50.1\% & 53.4\% & 54.0\% & 56.4\% \\ 
        \hline 
        \end{tabular}
        
        \begin{tabular}{|c||c|c|c|c|c|c|c|c|c|c|}
        \hline 
        Anzahl Iterationen & 6 & 7 & 8 & 9 & 10 & 1000\\ 
        \hline 
        J48 & / & / & / & / & / & / \\ 
        \hline 
        AdaBoost & 54.3\% & 55.9\% & 55.4\% & 56.9\% & 56.4\% & 60.0\% \\ 
        \hline 
        Bagging & 58.0\% & 59.0\% & 59.4\% & 59.6\% & 59.1\% & 62.1\% \\ 
        \hline 
        Random Forests & 56.7\% & 57.7\% & 57.5\% & 58.8\% & 58.8\% & 62.1\% \\ 
        \hline 
        \end{tabular} 
    \end{frame}   
    
    \begin{frame}[t]
    \frametitle{Aufgabe 7 - Ensemble-Lernen\\ Datensatz car}
        \begin{tabular}{|c||c|c|c|c|c|c|c|c|c|c|}
        \hline 
        Anzahl Iterationen & 1 & 2 & 3 & 4 & 5 \\ 
        \hline 
        J48 & 92.4\% & / & / & / & / \\ 
        \hline 
        AdaBoost & 92.4\% & 92.4\% & 94.7\% & 93.1\% & 95.5\% \\ 
        \hline 
        Bagging & 90.9\% & 91.6\% & 92.1\% & 92.5\% & 92.8\% \\ 
        \hline 
        Random Forests & 83.6\% & 87.5\% & 90.0\% & 91.4\% & 91.4\% \\ 
        \hline 
        \end{tabular}
        
        \begin{tabular}{|c||c|c|c|c|c|c|c|c|c|c|}
        \hline 
        Anzahl Iterationen & 6 & 7 & 8 & 9 & 10 & 1000\\ 
        \hline 
        J48 & / & / & / & / & / & / \\ 
        \hline 
        AdaBoost & 95.1\% & 96.1\% & 95.5\% & 96.2\% & 96.1\% & 97.2\% \\ 
        \hline 
        Bagging & 92.4\% & 92.5\% & 92.9\% & 93.2\% & 93.1\% & 93.7\% \\ 
        \hline 
        Random Forests & 91.9\% & 92.4\% & 92.6\% & 93.2\% & 93.6\% & 95.0\% \\ 
        \hline 
        \end{tabular} 
    \end{frame}
    
    \subsection{Vergleich und Interpretation}
    
    \begin{frame}[t]
    \frametitle{Aufgabe 7 - Ensemble-Lernen\\ Vergleich und Interpretation}
		\begin{itemize}
			\item Für alle Methoden gilt allgemein, dass mehr Iterationen eine höhere Genauigkeit bedeuten
			\item Bei geringen Iterationszahlen kann es bei AdaBoost und Bagging zu großen Schwankungen kommen
			\item Random Forests wird mit mehr Bäumen immer genauer, aber nie schlechter
			\item Random Forests ist immer am schnellsten, Bagging am zweitschnellsten und AdaBoost am langsamsten
			\begin{itemize}
				\item Random Forests ist zwischen 2x und 6x so schnell, je nach Datensatz
			\end{itemize}
			\item Für hohe Iterationszahlen sind alle Methoden besser als J48
			\item Die erzielte Accuracy und Verarbeitungsgeschwindigkeit scheinen stark datenabhängig zu sein
			\begin{itemize}
				\item Die Anzahl der Daten spielt dabei anscheinend eine untergeordnete Rolle
			\end{itemize}			
			\item Für schnelle Ergebnisse ist Random Forests am geeignetsten
		\end{itemize}
    \end{frame}
    
    \section{Aufgabe 8 - Pre-Processing}
    
    \subsection{Benutzte Datensätze}
    
    \begin{frame}[t]
    \frametitle{Aufgabe 8 - Pre-Processing\\ Benutzte Datensätze}
        \begin{itemize}
            \item autos
            \item iris
            \item sonar
        \end{itemize}
    \end{frame}
    
    \subsection{Erzielte Genauigkeiten}
    
    \begin{frame}[t]
    \frametitle{Aufgabe 8 - Pre-Processing\\ Erzielte Genauigkeiten}
        \begin{tabular}[htbp]{l||c|c|c}
            Datensatz & J48 Ursprünglich & J48 Diskretisiert \\
            \hline
            \hline
            autos & Acc. 82\%, Size 69 & Acc. 84\%, Size 103 \\
            \hline
            iris & Acc. 96\%, Size 9 & Acc. 94\%, Size 4 \\
            \hline
            sonar & Acc. 71\%, Size 35 & Acc. 80\%, Size 31 \\
        \end{tabular}
    \end{frame}
    
    \subsection{Vergleich und Interpretation}
    
    \begin{frame}[t]
    \frametitle{Aufgabe 8 - Pre-Processing\\ Vergleich und Interpretation}
        \begin{itemize}
            \item Genauigkeit
            \begin{itemize}
                \item J48 auf diskretisierten Daten ist im Schnitt besser als J48 auf den ursprünglichen Daten. Eine mögliche Erklärung wäre, dass die Daten nach dem Pre-Processing bereits einfacher und gruppiert sind, und dadurch leichter ein besseres generalisiertes Modell gelernt werden kann.
            \end{itemize}
        \end{itemize}
    \end{frame}

\begin{frame}
\frametitle{Abschlussüberblick}
\tableofcontents
\begin{center}
\textbf{\Large FRAGEN?}
\end{center}
\end{frame}

\begin{frame}
\frametitle{Gruppenmitglieder}
Joachim F. Brehmer-Moltmann, 1766932 \vfill
Jeannine Endreß, 1669152
\end{frame}

\end{document}
