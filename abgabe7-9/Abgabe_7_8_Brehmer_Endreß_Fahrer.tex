\documentclass[accentcolor=tud10b,colorbacktitle,inverttitle,landscape,german,presentation,t]{tudbeamer}
\usepackage{ngerman}
\usepackage[utf8]{inputenc}
\usepackage{graphicx}
\usepackage{multirow}
\providecommand\thispdfpagelabel[1]{}

\begin{document}

\title[MLDM: Projekt Aufg 7-8]{Maschinelles Lernen Symbolische Ansätze:\\ Projekt Aufgaben 7-8}
\subtitle{}

\author[brehmer\_endreß]{Joachim F. Brehmer-Moltmann, Jeannine Endreß}
%\institute[]{}

\date{\today}

\begin{titleframe}
\tableofcontents
\end{titleframe}

    \section{Aufgabe 7 - Ensemble-Lernen}
    
    \subsection{Benutzte Datensätze}
    
    \begin{frame}[t]
    \frametitle{Aufgabe 7 - Ensemble-Lernen\\ Benutzte Datensätze}
        \begin{itemize}
            \item labor
            \item yeast
            \item car
        \end{itemize}
    \end{frame}
    
    \subsection{Regulärer J48, Bagging und AdaBoost}
    
    \begin{frame}[t]
    \frametitle{Aufgabe 7 - Ensemble-Lernen\\ Regulärer J48, Bagging und AdaBoost}
        \begin{tabular}[htbp]{l||c|c|c|c|c}
            Datensatz & balance & breast & lenses & sonar & zoo \\
            \hline
            \hline
            Regulärer J48 & 76.6\% & 75.5\% & 83.3\% & 71.2\% & 92.1\% \\
            \hline
            Bagging J48 & 82.2\% & 73.4\% & 79.2\% & 74.5\% & 93.1\% \\
            \hline
            AdaBoost J48 & 78.9\% & 69.6\% & 70.8\% & 77.9\% & 95.0\% \\
            \hline
            Bagging RandomForest & 82.4\% & 69.2\% & 70.8\% & 86.5\% & 93.1\% \\
            \hline
            AdaBoost RandomForest & 78.4\% & 66.4\% & 79.2\% & 82.2\% & 90.1\%
        \end{tabular}
    \end{frame}
    
    \subsection{Vergleich und Interpretation}
    
    \begin{frame}[t]
    \frametitle{Aufgabe 7 - Ensemble-Lernen\\ Vergleich und Interpretation}
        \begin{itemize}
            \item Als Ensemble-Methode liefert Bagging insgesamt die besten Ergebnisse
            \item Dabei ist keine klare Struktur erkennbar, ob J48 oder RandomForest der bessere Lernalgorithmus für Bagging ist
            \item Bei AdaBoost lässt sich ebenfalls nicht definitiv festellen, ob J48 oder RandomForest besser geeignet wäre
            \item Auch wenn der reguläre J48 nicht überall schlechter ist, scheint insgesamt die Benutzung einer Ensemble-Methode sinnvoll zu sein, um bessere Genauigkeiten zu erzielen
            \item Alles in allem sieht es aber so aus, dass die erzielte Accuracy der einzelnen Algorithmen stark datenabhängig ist
        \end{itemize}
    \end{frame}
    
    \section{Aufgabe 8 - Pre-Processing}
    
    \subsection{Benutzte Datensätze}
    
    \begin{frame}[t]
    \frametitle{Aufgabe 8 - Pre-Processing\\ Benutzte Datensätze}
        \begin{itemize}
            \item 1985 Auto Imports Database
            \item Iris Plants Database
            \item Sonar, Mines vs. Rocks
        \end{itemize}
    \end{frame}
    
    \subsection{Erzielte Genauigkeiten}
    
    \begin{frame}[t]
    \frametitle{Aufgabe 8 - Pre-Processing\\ Erzielte Genauigkeiten}
        \begin{tabular}[htbp]{l||c|c|c}
            Datensatz & J48 Ursprünglich & J48 Diskretisiert \\
            \hline
            \hline
            autos & Acc. 82\%, Size 69 & Acc. 84\%, Size 103 \\
            \hline
            iris & Acc. 96\%, Size 9 & Acc. 94\%, Size 4 \\
            \hline
            sonar & Acc. 71\%, Size 35 & Acc. 80\%, Size 31 \\
        \end{tabular}
    \end{frame}
    
    \subsection{Vergleich und Interpretation}
    
    \begin{frame}[t]
    \frametitle{Aufgabe 8 - Pre-Processing\\ Vergleich und Interpretation}
        \begin{itemize}
            \item Genauigkeit
            \begin{itemize}
                \item J48 auf diskretisierten Daten ist im Schnitt besser als J48 auf den ursprünglichen Daten. Eine mögliche Erklärung wäre, dass die Daten nach dem Pre-Processing bereits einfacher und gruppiert sind, und dadurch leichter ein besseres generalisiertes Modell gelernt werden kann.
            \end{itemize}
        \end{itemize}
    \end{frame}

\begin{frame}
\frametitle{Abschlussüberblick}
\tableofcontents
\begin{center}
\textbf{\Large FRAGEN?}
\end{center}
\end{frame}

\begin{frame}
\frametitle{Gruppenmitglieder}
Joachim F. Brehmer-Moltmann, 1766932 \vfill
Jeannine Endreß, 1669152
\end{frame}

\end{document}
